\subsection{Généralités}

\begin{definition}
Une \tdef{onde progressive sinusoïdale (ou harmonique) unidimensionnelle} (ou \tdef{\abr{ops}}) est une onde progressive qui s'exprime si elle se propage vers les $x$ croissants (resp. décroissants) :
\[s(x, t) = A \cos(\omega t - kx + \varphi) \quad \text{(resp. } s(x, t) = A \cos(\omega t + kx + \varphi) \text{)}\]
avec $\omega$ la \tdef{pulsation}, $k$ le \tdef{nombre d'onde} et $\varphi$ la \tdef{phase à l'origine}.
\end{definition}

\begin{exemple}
On considère une corde supposée semi-infinie dans le sens des $x$ croissants, dont l'extrémité $\point{O}$ est fixée à un ressort vertical :

\begin{figure}[H]
\begin{center}
\begin{tikzpicture}[thick]
\node (O) at (0,0)    [inner sep=0pt, minimum size=0.4cm] {};
\node (A) at (0,1.5)  [inner sep=0pt, minimum size=0.4cm] {};


\draw [dashed,->] (6,0) -- (7,0) [right] node {$x$};
\draw [dashed,->] ($ (O) + (0,-0.2) $) -- ($ (A) + (0,0.5) $) [above] node {$s$};

\fill [bati] (A) +(-0.5,0) rectangle +(0.5,0.18);
\draw (A) +(-0.5,0) -- +(0.5,0);

\draw (A.south) -- (A.center);
\draw (O.north) -- (O.center) -- (6,0);
\draw [ressort={0.25}{1.2mm}] (A) -- (O);

\node             at (O) {$\bullet$};
\node [left]      at (O) {$\point{O}$};
\node [below=0.2] at (O) {$x = 0$};
	
\end{tikzpicture}
\end{center}
\end{figure}

\noindent Lorsque l'extrémité $\point{O}$ du ressort est écarté de sa position d'équilibre, il se met à vibrer sinusoïdalement (oscillateur harmonique). La perturbation en $x = 0$ s'écrit alors :
\[s(0, t) = A \cos(\omega t + \varphi)\]
En réponse à cette excitation, une onde transverse, que l'on suppose non amortie et de célerité $c$ constante, se propage vers les $x$ croissants :
\[s(x, t) = f(x - ct) = s\left(0, t - \frac{x}{c}\right) = A \cos\left(\omega t - \frac{\omega}{c} x + \varphi\right)\]
L'onde ainsi créée est une \abr{ops}.
\end{exemple}

\begin{propriete}
Une \abr{ops} présente une \tdef{double périodicité} :

\begin{itemize}
\item une \imp{périodicité temporelle}, caractérisée par la \tdef{période} $T = \frac{2\pi}{\omega}$ ou bien la \tdef{fréquence} $\nu = \frac{1}{T}$;

\item une \imp{périodicité spatiale}, caractérisée par la \tdef{longueur d'onde} $\lambda = \frac{2\pi}{k}$.
\end{itemize}
\end{propriete}

\begin{propriete}
Les grandeurs caractéristiques d'une \abr{ops} sont reliées par les \tdef{relations de dispersion} (équivalentes entre elles) :

\begin{center}
\begin{enumerate*}[label=(\roman*), itemjoin=\qquad]
\item $k = \frac{\omega}{c}$
\item $c = \lambda \nu$
\item $\lambda = cT$
\end{enumerate*}
\end{center}
\end{propriete}



\subsection{Déphasage}

\begin{definition}
Deux \imp{signaux sinusoïdaux} sont \tdef{synchrones} s'ils ont même pulsation $\omega$ ou, de manière équivalente, même fréquence $\nu$ ou même période $T$.
\end{definition}

\begin{definition}
Le \tdef{déphasage} $\Delta\varphi$ de deux \imp{signaux sinusoïdaux synchrones}
\[\begin{cases}
s_1(t) = A_1 \cos(\phi_1(t)) & \text{avec } \phi_1(t) = \omega t + \varphi_1\\
s_2(t) = A_2 \cos(\phi_2(t)) & \text{avec } \phi_2(t) = \omega t + \varphi_2
\end{cases}\]
est l'unique réel de $]-\pi, \pi]$ tel que :
\[\Delta\varphi \equiv \phi_2(t) - \phi_1(t) \equiv \varphi_2 - \varphi_1 \pmod{2\pi}\]
\end{definition}

\begin{propriete}
Soit $s_1(t)$ et $s_2(t)$ deux \imp{signaux sinusoïdaux synchrones} de pulsation $\omega$, $t_1$ et $t_2$ deux instants tels que :
\[s_1(t_1) = s_2(t_2) \quad \text{et} \quad \frac{\dif s_1}{\dif t}(t_1) = \frac{\dif s_2}{\dif t}(t_2)\]
Alors :
\[\Delta\varphi \equiv -\omega(t_2 - t_1) \pmod{2\pi}\]
\end{propriete}

\begin{vocabulaire}
Soit $s_1(t)$ et $s_2(t)$ deux \imp{signaux sinusoïdaux synchrones}, on note $\Delta\varphi$ leur déphasage. On dit que :

\begin{itemize}
\item $s_2$ est en \tdef{avance de phase} (resp. \tdef{retard de phase}) sur $s_1$ lorsque $\Delta\varphi > 0$ (resp. $\Delta\varphi < 0$);

\item $s_1$ et $s_2$ sont en \tdef{phase} lorsque $\Delta\varphi = 0$;

\item $s_1$ et $s_2$ sont en \tdef{opposition de phase} lorsque $\Delta\varphi = \pi$;

\item $s_1$ et $s_2$ sont en \tdef{quadrature} lorsque $\Delta\varphi = \pm\frac{\pi}{2}$.
\end{itemize}
\end{vocabulaire}



\subsection{De l'\abr{ops} à la vibration quelconque}

\begin{propriete}[admis]
Une vibration quelconque peut toujours être exprimée comme une somme d'\abr{ops} en utilisant la \tdef{transformation de Fourier}.
\end{propriete}