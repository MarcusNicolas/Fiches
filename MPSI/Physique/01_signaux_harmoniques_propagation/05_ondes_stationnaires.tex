\subsection{Ondes stationnaires sinusoïdales}

\begin{definition}
Une \tdef{onde stationnaire unidimensionnelle} est une onde unidimensionnelle pouvant s'exprimer sous la forme :
\[s(x, t) = f(x) g(t)\] 
\end{definition}

\begin{definition}
Une \tdef{onde stationnaire sinusoïdale unidimensionnelle} (ou \tdef{\abr{oss}}) est une onde stationnaire unidimensionnelle dont les composantes temporelle et spatiale sont sinusoïdales :
\[s(x, t) = C \cos(kx + \varphi) \cos(\omega t + \psi)\]
\end{definition}

\begin{propriete}
La superposition de deux \imp{\abr{ops} synchrones} de même \imp{amplitude} et \imp{contre-propageantes} donne naissance à une \abr{oss}.
\end{propriete}

\begin{exemple}
Considérons une corde semi-infinie dans la direction des $x$ décroissants, fixée en son extrémité au point $\point{O}$, sur laquelle on envoie un \abr{ops} se dirigeant vers les $x$ croissants $A \cos(\omega t - kx)$. On suppose que l'onde se réfléchit en $x = 0$, donnant naissance à une \abr{ops} régressive de même pulsation $\omega$ et nombre d'onde $k$, dont on montre qu'elle s'écrit $-A \cos(\omega t + kx)$. Finalement, d'après le principe de superposition, les oscillations de la corde s'écrivent :
\[s(x, t) = A \cos(\omega t - kx) - A \cos(\omega t + kx) = 2A \sin(kx) \sin(\omega t)\]
\end{exemple}

\begin{vocabulaire}
En un point d'abscisse $x$, les oscillations au cours du temps d'une \abr{oss} $s(x, t) = C \cos(kx + \varphi) \cos(\omega t + \psi)$ sont d'amplitude $C \abs{\cos(kx + \varphi)}$. On dit que $x$ est :
\begin{itemize}
\item un \tdef{ventre} de vibration si $\cos(kx + \varphi) = \pm 1$ (amplitude maximale);

\item un \tdef{n\oe{}ud} de vibration si $\cos(kx + \varphi) = 0$ (amplitude nulle);
\end{itemize}
\end{vocabulaire}

\begin{propriete}
Deux n\oe{}uds ou ventres successifs sont distants de $\frac{\lambda}{2}$ :

\begin{figure}[H]
\begin{center}
\begin{tikzpicture}[thick, domain=-8.5:0]

\draw [dashed, ->] (-9.5,0) -- (1,0) node [right] {$x$};
\draw [dashed, ->] (0,0) -- (0,1) node [above] {$s$};

\foreach \i in {-2,...,2}
\draw [black, line width=0.25pt*(4+\i), samples=200] plot (\x,{\i/3 * sin((\x) * pi/2 r)});

\fill [bati] (0,-0.5) rectangle (0.18,0.5);
\draw (0,-0.5) -- (0,0.5);

\draw [<->] (-7,2/3+0.2) -- node [above] {$\frac{\lambda}{2}$} +(2,0);
\draw [<-] (-6,-0.1) -- +(0,-0.3-2/3) node [below] {n\oe{}ud};
\draw [<-] (-3,-0.1-2/3) -- +(0,-0.3) node [below] {ventre};
\end{tikzpicture}
\end{center}
\end{figure}
\end{propriete}



\subsection{Modes propres d'une cavité}

\begin{definition}
Les \tdef{modes propres} de vibration d'une cavité sont les \abr{oss} susceptibles d'y perdurer.
\end{definition}

\begin{propriete}
Les longueurs d'onde $\lambda_n$ (et donc les fréquences $\nu_n$) accessibles aux modes propres d'une cavité de longueur $L$ sont \imp{quantifiées}, du fait des \imp{conditions aux limites} :
\[\lambda_n = \frac{2L}{n} \quad \left(\text{donc } k_n = n \frac{\pi}{L} \right) \quad \text{et} \quad \nu_n = n \frac{c}{2L} \quad \left(\text{donc } \omega_n = n \frac{\pi c}{L} \right)\]

\noindent On peut donc représenter l'allure des premiers modes propres $n$ de vibration :

\begin{figure}[H]
\begin{center}
\begin{tikzpicture}[thick, domain=0:8]
\draw [dashed, ->] (0,0) -- (0,1) node [above] {$s$};
\draw [dashed, ->] (0,0) -- (9,0) node [right] {$x$};

\foreach \n in {1,2,3}
{
\begin{scope}[yshift=-(\n-1)*1.5cm]

\foreach \i in {-2,...,2}
\draw [black, line width=0.25pt*(4+\i), samples=200] plot (\x,{\i/3 * sin((\x) * \n*pi/8 r)});

\fill [bati] (-0.18,-0.5) rectangle (0,0.5);
\fill [bati] (8,-0.5) rectangle (8.18,0.5);
\draw (0,-0.5) -- (0,0.5);
\draw (8,-0.5) -- (8,0.5);

\node at (-1.5,0) {$n=\n$};
\end{scope}
}
\end{tikzpicture}
\end{center}
\end{figure}
\end{propriete}

\begin{experience}
En pratique on peut utiliser une \tdef{corde de Melde} (et un stroboscope) pour visualiser les modes propres :

\begin{figure}[H]
\begin{center}
\begin{tikzpicture}[thick, domain=0:8]
\foreach \i in {-1,...,1}
\draw [black, line width=0.5pt*(2+\i), samples=200] plot (\x,{\i/3 * sin((\x) * pi/4 r)});

\node (vibreur) at (9,0) [rectangle, schema, minimum width=2cm, minimum height=4cm/3] {vibreur};
\node (masse) at (-0.25,-1.5) [rectangle, schema=1pt, minimum width=0.3cm, minimum height=0.5cm] {};

\draw (-0.25,-0.25) -- (masse);

\node (poulie) at ({-sqrt(0.25^2-0.25^2},-0.25) [circle, schema, minimum size=0.6cm] {};

\draw [->] (0.7,-0.8) node [right] {poulie}        -- (poulie);
\draw [->] (0.7,-1.5) node [right] {masse pesante} -- (masse);
\end{tikzpicture}
\end{center}
\end{figure}
\end{experience}

\begin{propriete}[admis]
Une vibration quelconque \imp{perdurant} dans une \imp{cavité} peut toujours être exprimée comme une somme de modes propres.
\end{propriete}