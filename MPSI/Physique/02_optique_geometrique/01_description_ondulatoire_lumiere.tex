\subsection{Spectres d'émission}

\begin{definition}
Le \tdef{spectre d'émission} d'une source lumineuse est l'ensemble des fréquences $\nu$ (ou des longueurs d'onde $\lambda_0$ dans le vide correspondantes) contenues dans le rayonnement émis par cette source.
\end{definition}

\begin{vocabulaire}
On distingue trois grandes catégories de spectres :
\begin{itemize}
\item Le \tdef{spectre d'émission continu} se présente sous la forme d'une bande colorée ininterrompue. Il est caractéristique des \imp{corps chauds et denses}. C'est par exemple le cas du filament d'une lampe à incandescence, ou bien de la surface des étoiles.

\begin{figure}[H]
\begin{center}
\pgfspectra[width=0.8\columnwidth,axis,axis step=40]
\end{center}
\end{figure}

\item Un \tdef{spectre d'émission de raies} ne contient qu'un nombre restreint de radiations quasi-monochromatiques appelées \tdef{raies}. Il est émis par un \imp{gaz chaud et à basse pression}. En pratique, les lampes à décharge contenant un gaz ou des vapeurs métalliques donnent ce type de spectre.

\begin{figure}[H]
\begin{center}
\pgfspectra[width=0.8\columnwidth,axis,axis step=40,element=He]

\captionsetup{labelformat=empty}
\caption{Spectre d'émission de l'atome d'hélium}
\end{center}
\end{figure}

\item Le \tdef{spectre d'émission monochromatique} ne contient qu'une seule raie. Une source associée à ce type de spectre est dite \tdef{monochromatique}.

\begin{figure}[H]
\begin{center}
\pgfspectra[width=0.8\columnwidth,axis,axis step=40,lines={633}]

\captionsetup{labelformat=empty}
\caption{Spectre d'émission d'un \abr{laser} hélium-néon}
\end{center}
\end{figure}
\end{itemize}
\end{vocabulaire}



\subsection{Propagation}

\begin{definition}
Un milieu est :
\begin{itemize}
\item \tdef{transparent} lorsque l'extinction de la lumière y est négligeable;
\item \tdef{homogène} si ses propriétés physiques sont identiques en tout point;
\item \tdef{isotrope} si ses propriétés physiques ne dépendent pas de la direction.
\end{itemize}
Quand un milieu a toutes ces propriétés, on parle de \tdef{milieu \abr{thi}} (ou \tdef{\abr{mthi}}).
\end{definition}

\begin{definition}
On note $v(\lambda_0) \leq c$ la vitesse de propagation d'une onde électromagnétique de longueur d'onde $\lambda_0$ dans le vide dans un \abr{mthi}. On appelle alors \tdef{indice de réfraction} de ce milieu pour $\lambda_0$ le scalaire :
\[n(\lambda_0) = \frac{c}{v(\lambda_0)} \quad \text{avec } n(\lambda_0) \geq 1\] 
Le milieu est dit \tdef{dispersif} lorsque $n$ dépend de $\lambda_0$.
\end{definition}

\begin{remarque}
Tous les milieux sauf le vide sont plus ou moins dispersifs.
\end{remarque}

\begin{propriete}[admis]
Les \abr{mthi} ont un indice de réfraction qui suit généralement la \tdef{loi de Cauchy} :
\[n(\lambda_0) = A + \frac{B}{{\lambda_0}^2} \quad \text{avec } A, B \geq 0 \text{ des constantes du milieu}\]
\end{propriete}

\begin{propriete}[admis]
Dans un \abr{mthi}, les grandeurs $\lambda$ et $\nu$ d'une onde électromagnétique sont reliées par la \tdef{relation de dispersion} $v = \lambda \nu$.
\end{propriete}

\begin{propriete}
On considère une onde électromagnétique de longueurs d'onde $\lambda_0$ dans le vide et $\lambda$ dans un \abr{mthi}. Alors :
\[\lambda = \frac{\lambda_0}{n(\lambda_0)} \quad \text{et donc } \lambda \leq \lambda_0\]
\end{propriete}

\begin{remarque}
La couleur perçue d'un rayonnement visible dépend seulement de sa fréquence $\nu$ donc de sa longueur d'onde dans le vide $\lambda_0$, pas de sa longueur d'onde dans le milieu $\lambda$.
\end{remarque}