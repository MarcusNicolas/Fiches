\subsection{Systèmes optiques}

\begin{definition}
Un \tdef{système optique} $\mathscr{S}$ est un ensemble de milieux séparés par des surfaces réfractantes (dioptres) ou réfléchissantes (miroirs).
\end{definition}

\begin{definition}
On appelle \tdef{rayon incident} (resp. \tdef{émergent}) un rayon lumineux arrivant sur le  (resp. ressortant du) système optique dans le \imp{sens de propagation de la lumière} :

\begin{figure}[H]
\begin{center}
\begin{tikzpicture}[thick, use optics]
\draw [blue, ->-] (-2,-1) node [left] {rayon incident} -- (-0.7,-0.5);
\draw [red,  -<-] (2,1)   node [right] {rayon émergent} -- (0.7,0.5);


\draw (-0.4,-1.2) -- (-0.7,-1.2) -- (-0.7,1.2) -- (-0.4,1.2);
\draw  (0.4,-1.2) --  (0.7,-1.2) --  (0.7,1.2) --  (0.4,1.2);


\draw [blue, dashed] (-0.7,-0.5) -- (2,0.54);
\draw [red,  dashed] (0.7,0.5)   -- (-2,-0.54);

\draw [<-] (1.5,0.35) -- (2,-0.5) node [right] {prolongement};
\node [below] at (0,-1.2) {$\mathscr{S}$};
\end{tikzpicture}
\end{center}
\end{figure}
\end{definition}



\subsection{Objets et images}

\begin{definition}
On appelle \tdef{objet ponctuel} (resp. \tdef{image ponctuelle}) ou \tdef{point objet} (resp. \tdef{point image}) pour un certain système optique l'intersection des \imp{rayons incidents} (resp. \imp{émergents}) (caractère \tdef{réel}) ou de leurs prolongements (caractère \tdef{virtuel}).

\begin{figure}[H]
\begin{center}
\begin{tikzpicture}[thick, use optics]
\coordinate (A)  at (-2,0.5);
\coordinate (B') at (-2.3,-0.5);


\draw [red, ->-] (A) -- +(1.3,0.4);
\draw [red, ->-] (A) -- +(1.3,0);
\draw [red, ->-] (A) -- +(1.3,-0.4);
\node [left] at (A) {$\point{A}$};


\draw [blue, ->-] (B') ++(3,0.4)  -- +(1.5,0.2);
\draw [blue, ->-] (B') ++(3,0)    -- +(1.5,0);
\draw [blue, ->-] (B') ++(3,-0.4) -- +(1.5,-0.2);

\node [left] at (B') {$\point{B'}$};


\draw (-0.4,-1.2) -- (-0.7,-1.2) -- (-0.7,1.2) -- (-0.4,1.2);
\draw  (0.4,-1.2) --  (0.7,-1.2) --  (0.7,1.2) --  (0.4,1.2);
\node [below] at (0,-1.2) {$\mathscr{S}$};


\draw [blue, dashed] (B') -- +(3,0.4);
\draw [blue, dashed] (B') -- +(3,0);
\draw [blue, dashed] (B') -- +(3,-0.4);
\end{tikzpicture}
\end{center}

\captionsetup{labelformat=empty}
\caption{Ici, $\point{A}$ est un objet réel pour $\mathscr{S}$ et $\point{B'}$ est une image virtuelle}
\end{figure}
\end{definition}

\begin{remarque}
Les objets et images virtuels ne sont \imp{pas visibles par un capteur} car l'énergie lumineuse ne s'y concentre pas. L'\oe{}il étant un système optique à part entière, il est potentiellement capable d'observer tout type d'objet ou d'image pour un autre système optique.
\end{remarque}

\begin{definition}
Un \tdef{objet étendu} (resp. \tdef{image étendue}) est un ensemble de \imp{points objets} (resp. \imp{images}) conjoints.

\begin{figure}[H]
\begin{center}
\begin{tikzpicture}[thick, use optics]

\coordinate (A') at (0.1,-0.8);
\coordinate (B') at (0.1,0.8);

\begin{scope}
	\clip (0.7,-1.5) rectangle (2,1.5);
	
	\draw [blue, ->-]  (A') -- +(2,0.4);
	\draw [blue, ->-]  (A') -- +(2,0);
	\draw [blue, ->-]  (A') -- +(2,-0.4);
	
	\draw [red, ->-]  (B') -- +(2,0.4);
	\draw [red, ->-]  (B') -- +(2,0);
	\draw [red, ->-]  (B') -- +(2,-0.4);
\end{scope}

\begin{scope}
	\clip (-0.7,-1.5) rectangle (0.7,1.5);
	
	\draw [blue, dashed]  (A') -- +(2,0.4);
	\draw [blue, dashed]  (A') -- +(2,0);
	\draw [blue, dashed]  (A') -- +(2,-0.4);
	
	\draw [red, dashed]  (B') -- +(2,0.4);
	\draw [red, dashed]  (B') -- +(2,0);
	\draw [red, dashed]  (B') -- +(2,-0.4);
\end{scope}

\node [left] at (A') {$\point{A'}$};
\node [left] at (B') {$\point{B'}$};
\draw [dashed, ->] (A') -- (B');


\draw (-0.4,-1.2) -- (-0.7,-1.2) -- (-0.7,1.2) -- (-0.4,1.2);
\draw  (0.4,-1.2) --  (0.7,-1.2) --  (0.7,1.2) --  (0.4,1.2);
\node [below] at (0,-1.2) {$\mathscr{S}$};
\end{tikzpicture}
\end{center}

\captionsetup{labelformat=empty}
\caption{Ici, $\point{A'}\point{B'}$ est une image étendue, virtuelle (donc en pointillés)}
\end{figure}
\end{definition}



\subsection{Stigmatisme}

\begin{definition}
Un système optique $\mathscr{S}$ est dit \tdef{rigoureusement stigmatique} pour un couple de points $\point{A}$ et $\point{A'}$ si les \imp{rayons incidents} issus du \imp{point objet} $\point{A}$ ne donnent lieu, après avoir traversé $\mathscr{S}$, qu'à un unique \imp{point image}, $\point{A'}$ :

\begin{figure}[H]
\begin{center}
\begin{tikzpicture}[thick, use optics]
\coordinate (A)  at (-2,0);
\coordinate (A') at (2,0);


\draw [red, ->-] (A) -- +(1.3,0.4);
\draw [red, ->-] (A) -- +(1.3,0);
\draw [red, ->-] (A) -- +(1.3,-0.4);
\node [left] at (A) {$\point{A}$};

\draw [red, -<-] (A') -- +(-1.3,0.4);
\draw [red, -<-] (A') -- +(-1.3,0);
\draw [red, -<-] (A') -- +(-1.3,-0.4);
\node [right] at (A') {$\point{A'}$};


\draw (-0.4,-1.2) -- (-0.7,-1.2) -- (-0.7,1.2) -- (-0.4,1.2);
\draw  (0.4,-1.2) --  (0.7,-1.2) --  (0.7,1.2) --  (0.4,1.2);
\node [below] at (0,-1.2) {$\mathscr{S}$};
\end{tikzpicture}
\end{center}
\end{figure}

\noindent On dit alors que $\point{A}$ et $\point{A'}$ sont \tdef{conjugués} par $\mathscr{S}$ (ou que $\point{A'}$ est l'image de $\point{A}$ par $\mathscr{S}$), ce que l'on note $\point{A} \stackrel{\mathscr{S}}{\longmapsto} \point{A'}$.
\end{definition}

\begin{remarque}
Lorsque l'image d'un point $\point{A}$ par $\mathscr{S}$ n'est pas rigoureusement ponctuelle mais est une tâche de faible dimension, on parle de \tdef{stigmatisme approché} (et on continue de noter $\point{A} \stackrel{\mathscr{S}}{\longmapsto} \point{A'}$).
\end{remarque}