\subsection{Objets et images à l'infini}

\begin{definition}
Un \tdef{objet ponctuel à l'infini} (resp. \tdef{image ponctuelle à l'infini}) est un faisceau de \imp{rayons incidents} (resp. \imp{émergents}) parallèles, que l'on repère à l'aide de son inclinaison par rapport à l'axe optique.

\begin{figure}[H]
\begin{center}
\begin{tikzpicture}[thick, use optics]
\coordinate (A) at (-3,0);

\draw (-4,0) -- (4,0) node [right] {$\droite{\Delta}$};

\draw [blue, ->-] (-3.5,0.7)                                               -- +(2.8,0);
\draw [blue, ->-] (-3.5,0) node [above left, black] {$\point{A}_{\infty}$} -- +(2.8,0);
\draw [blue, ->-] (-3.5,-0.7)                                              -- +(2.8,0);

\draw [red, ->-]  (0.7,0.7)  -- +(10:2.84);
\draw [red, ->-]  (0.7,0)    -- +(10:2.84) node [right, black] {$\point{B'}_{\infty}$};
\draw [red, ->-]  (0.7,-0.7) -- +(10:2.84);

\draw [->] (0.7,0) +(2,0) arc (0:10:2) node [midway, right] {$\beta'$};


\draw (-0.4,-1.2) -- (-0.7,-1.2) -- (-0.7,1.2) -- (-0.4,1.2);
\draw  (0.4,-1.2) --  (0.7,-1.2) --  (0.7,1.2) --  (0.4,1.2);
\node [below] at (0,-1.2) {$\mathscr{S}$};

\draw [->] (0.4,1.5) arc (0:90:0.4);
\node at (0,1.5) {$+$};
\end{tikzpicture}
\end{center}

\captionsetup{labelformat=empty}
\caption{$\point{A}_{\infty}$ à l'infini sur l'axe optique, $\point{B'}_{\infty}$ est situé à l'infini hors axe optique ($\beta' \neq 0$)}

\end{figure}
\end{definition}

\begin{definition}
Lorsqu'un objet étendu $\point{A}_{\infty}\point{B}_{\infty}$ \imp{plan} et \imp{centré sur l'axe optique} est \imp{situé à l'infini}, on dit qu'il possède un \tdef{diamètre angulaire} (ou \tdef{diamètre apparent}) $2\alpha$, où le \tdef{rayon angulaire} (ou \tdef{rayon apparent}) $\alpha$ désigne l'angle des faisceaux issus de $\point{A}_{\infty}$ et $\point{B}_{\infty}$ avec l'axe optique :

\begin{figure}[H]
\begin{center}
\begin{tikzpicture}[thick, use optics]

\draw (-4,0) -- (4,0) node [right] {$\droite{\Delta}$};

\draw [-] (-0.7,0) +(-2,0) arc (180:170:2) node [midway, left] {$\alpha$};

\draw [red, -<-]  (-0.7,0.7)  -- +(170:2.84);
\draw [red, -<-]  (-0.7,0.35) -- +(170:2.84) node [left, black] {$\point{A}_{\infty}$};
\draw [red, -<-]  (-0.7,0)     -- +(170:2.84);

\draw [blue, -<-] (-0.7,-0.7)  -- +(-170:2.84);
\draw [blue, -<-] (-0.7,-0.35) -- +(-170:2.84) node [left, black] {$\point{B}_{\infty}$};
\draw [blue, -<-] (-0.7,0)     -- +(-170:2.84);


\draw (-0.4,-1.2) -- (-0.7,-1.2) -- (-0.7,1.2) -- (-0.4,1.2);
\draw  (0.4,-1.2) --  (0.7,-1.2) --  (0.7,1.2) --  (0.4,1.2);
\node [below] at (0,-1.2) {$\mathscr{S}$};
\end{tikzpicture}
\end{center}
\end{figure}
\end{definition}


\subsection{Foyers et plans focaux}

\begin{definition}
Le \tdef{foyer objet} $\point{F}$ (resp. \tdef{foyer image} $\point{F'}$) est le \imp{point objet} (resp. \imp{point image}) dont le conjugué se trouve à l'infini sur  $\droite{\Delta}$ :
\[\point{F} \stackrel{\mathscr{S}}{\longmapsto} \point{A'}_{\infty} \in \droite{\Delta} \qquad \text{et} \qquad \point{A}_{\infty} \in \droite{\Delta} \stackrel{\mathscr{S}}{\longmapsto} \point{F'}\]
On appelle \tdef{plan focal objet} (resp. \tdef{plan focal image}) le plan transversal passant par $\point{F}$ (resp. par $\point{F'}$).

\begin{figure}[H]
\begin{center}
\begin{tikzpicture}[thick, use optics]
\coordinate (O1) at (-3,0);
\coordinate (O2) at (3,0);
\coordinate (F)  at (-2,0);
\coordinate (F') at (2,0);

% Système 1

\draw (O1) +(-2.5,0) -- +(2.5,0) node [right] {$\droite{\Delta}$};

\draw [red, ->-] ($ (O1) + (F) $) -- +(1.3,0.5);
\draw [red, ->-] ($ (O1) + (F) $) -- +(1.3,0);
\draw [red, ->-] ($ (O1) + (F) $) -- +(1.3,-0.5);

\draw [red, ->-] ($ (O1) + (F) + (2.7,0.5)  $) node [above right, black] {$\point{A'}_{\infty} \in \droite{\Delta}$} -- +(1.3,0);
\draw [red, ->-] ($ (O1) + (F) + (2.7,0 )   $)                                                                 -- +(1.3,0);
\draw [red, ->-] ($ (O1) + (F) + (2.7,-0.5) $)                                                                 -- +(1.3,0);

\draw [dashed] ($ (O1) + (F) $) +(0,-1.2) -- +(0,1.2) node [above] {plan focal objet};
\node [above left] at ($ (O1) + (F) $) {$\point{F}$};

\draw (O1) +(-0.4,-1.2) -- +(-0.7,-1.2) -- +(-0.7,1.2) -- +(-0.4,1.2);
\draw  (O1) +(0.4,-1.2) -- +(0.7,-1.2)  -- +(0.7,1.2)  -- +(0.4,1.2);
\node [below] at ($ (O1) + (0,-1.2) $) {$\mathscr{S}$};


% Système 2

\draw (O2) +(-2.5,0) -- +(2.5,0) node [right] {$\droite{\Delta}$};

\draw [red, -<-] ($ (O2) + (F') $) -- +(-1.3,0.5);
\draw [red, -<-] ($ (O2) + (F') $) -- +(-1.3,0);
\draw [red, -<-] ($ (O2) + (F') $) -- +(-1.3,-0.5);

\draw [red, -<-] ($ (O2) + (F') + (-2.7,0.5)  $) node [above left, black] {$\point{A}_{\infty} \in \droite{\Delta}$} -- +(-1.3,0);
\draw [red, -<-] ($ (O2) + (F') + (-2.7,0)    $)                                                                 -- +(-1.3,0);
\draw [red, -<-] ($ (O2) + (F') + (-2.7,-0.5) $)                                                                 -- +(-1.3,0);

\draw [dashed] ($ (O2) + (F') $) +(0,-1.2) -- +(0,1.2) node [above] {plan focal image};
\node [above right] at ($ (O2) + (F') $) {$\point{F'}$};

\draw (O2) +(-0.4,-1.2) -- +(-0.7,-1.2) -- +(-0.7,1.2) -- +(-0.4,1.2);
\draw  (O2) +(0.4,-1.2) -- +(0.7,-1.2)  -- +(0.7,1.2)  -- +(0.4,1.2);
\node [below] at ($ (O2) + (0,-1.2) $) {$\mathscr{S}$};
\end{tikzpicture}
\end{center}
\end{figure}
\end{definition}

\begin{propriete}
Tout rayon incident passant par $\point{F}$ (resp. parallèle à $\droite{\Delta}$) émerge du système optique parallèment à $\droite{\Delta}$ (resp. en passant par $\point{F'}$).
\end{propriete}

\begin{propriete}[admis]
Tout \imp{système centré} étudié dans les \imp{conditions de Gauss} possède un foyer image et un foyer objet.
\end{propriete}

\begin{vocabulaire}
Dans le cas particulier où $\point{F}$ et $\point{F'}$ sont rejetés à l'infini, le système est dit \tdef{afocal}.
\end{vocabulaire}

\begin{definition}
On appelle \tdef{foyer secondaire objet} $\point{\phi}$ (resp. \tdef{foyer secondaire image} $\point{\phi'}$) tout point du \imp{plan focal objet} (resp. \imp{plan focal image}).
\end{definition}

\begin{propriete}
Par aplanétisme, le conjugué d'un \imp{foyer secondaire objet} $\point{\phi}$ (resp. \imp{foyer secondaire image} $\point{\phi'}$) est un point situé à l'infini :
\[\point{\phi} \stackrel{\mathscr{S}}{\longmapsto} \point{A'}_{\infty} \qquad \text{(resp. } \point{A}_{\infty} \stackrel{\mathscr{S}}{\longmapsto} \point{\phi'} \text{)}\]

\begin{figure}[H]
\begin{center}
\begin{tikzpicture}[thick, use optics]
\coordinate (O1)   at (-3,0);
\coordinate (O2)   at (3,0);
\coordinate (F)  at (-2,0);
\coordinate (F') at (2,0);
\coordinate (phi)  at (-2,0.5);
\coordinate (phi') at (2,-0.5);

% Système 1

\draw (O1) +(-2.5,0) -- +(2.5,0) node [right] {$\droite{\Delta}$};

\draw [red, ->-] ($ (O1) + (phi) $) -- +(1.3,0);
\draw [red, ->-] ($ (O1) + (phi) $) -- +(1.3,-0.5);
\draw [red, ->-] ($ (O1) + (phi) $) -- +(1.3,-1);

\draw [red, ->-] ($ (O1) + (phi) + (2.7,0)    $) -- +(1.3,-0.5);
\draw [red, ->-] ($ (O1) + (phi) + (2.7,-0.5) $) -- +(1.3,-0.5) node [right, black] {$\point{A'}_{\infty}$};
\draw [red, ->-] ($ (O1) + (phi) + (2.7,-1)   $) -- +(1.3,-0.5);

\draw [dashed] ($ (O1) + (F) $) +(0,-1.2) -- +(0,1.2) node [above] {plan focal objet};
\node [below left] at ($ (O1) + (F) $) {$\point{F}$};
\node [left] at ($ (O1) + (phi) $) {$\point{\phi}$};

\draw (O1) +(-0.4,-1.2) -- +(-0.7,-1.2) -- +(-0.7,1.2) -- +(-0.4,1.2);
\draw  (O1) +(0.4,-1.2) -- +(0.7,-1.2)  -- +(0.7,1.2)  -- +(0.4,1.2);
\node [below] at ($ (O1) + (0,-1.2) $) {$\mathscr{S}$};


% Système 2

\draw (O2) +(-2.5,0) -- +(2.5,0) node [right] {$\droite{\Delta}$};

\draw [red, -<-] ($ (O2) + (phi') $) -- +(-1.3,0);
\draw [red, -<-] ($ (O2) + (phi') $) -- +(-1.3,0.5);
\draw [red, -<-] ($ (O2) + (phi') $) -- +(-1.3,1);

\draw [red, -<-] ($ (O2) + (phi') + (-2.7,0)   $) -- +(-1.3,0.5);
\draw [red, -<-] ($ (O2) + (phi') + (-2.7,0.5) $) -- +(-1.3,0.5) node [left, black] {$\point{A}_{\infty}$};
\draw [red, -<-] ($ (O2) + (phi') + (-2.7,1)   $) -- +(-1.3,0.5);

\draw [dashed] ($ (O2) + (F') $) +(0,-1.2) -- +(0,1.2) node [above] {plan focal image};
\node [above right] at ($ (O2) + (F') $) {$\point{F'}$};
\node [right] at ($ (O2) + (phi') $) {$\point{\phi'}$};

\draw (O2) +(-0.4,-1.2) -- +(-0.7,-1.2) -- +(-0.7,1.2) -- +(-0.4,1.2);
\draw  (O2) +(0.4,-1.2) -- +(0.7,-1.2)  -- +(0.7,1.2)  -- +(0.4,1.2);
\node [below] at ($ (O2) + (0,-1.2) $) {$\mathscr{S}$};
\end{tikzpicture}
\end{center}
\end{figure}
\end{propriete}