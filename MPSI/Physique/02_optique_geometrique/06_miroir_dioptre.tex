\subsection{Miroir plan}

\begin{definition}
On modélise un \tdef{miroir plan} $\mathscr{M}$ par une surface plane parfaitement réflechissante :

\begin{figure}[H]
\begin{center}
\begin{tikzpicture}[thick, use optics]

\draw [red, ->-] (-2,1) -- (0,0);
\draw [red, ->-] (0,0)  -- (-2,-1);

\draw [dashed] (0,0) -- (-2.5,0) node [left] {$\droite{\mathcal{N}}$};
\draw (-0.15,0) |- (0,0.15);

\node [mirror, object height=2.4cm] (miroir) at (0,0) {};
\node [below=0.15] at (miroir.south) {$\mathscr{M}$};
\end{tikzpicture}
\end{center}
\end{figure}
\end{definition}

\begin{propriete}
Un \imp{miroir plan} est rigoureusement stigmatique et aplanétique.
\end{propriete}

\begin{propriete}
Soit $\point{A}$ un \imp{point objet} d'image $\point{A'}$ par $\mathscr{M}$ et de projeté orthogonal $\point{H}$ sur le plan de $\mathscr{M}$ :

\begin{figure}[H]
\begin{center}
\begin{tikzpicture}[thick, use optics]
\coordinate (A)  at (-2,0);
\coordinate (A') at (2,0);
\coordinate (H)  at (0,0);

\draw [red, ->-] (A) -- ($ (H) + (0,0) $);
\draw [red, ->-] (A) -- ($ (H) + (0,0.75) $);
\draw [red, ->-] (A) -- ($ (H) + (0,-0.75) $);

\draw [red, ->-] (A)           -- +(-1,0);
\draw [red, ->-] (H) +(0,0.75)  -- ($ (A) + (0,1.5) $);
\draw [red, ->-] (H) +(0,-0.75) -- ($ (A) +(0,-1.5) $);

\node [mirror, object height=2.4cm] (miroir) at (0,0) {};
\node [below=0.15] at (miroir.south) {$\mathscr{M}$};


\draw [red, dashed] (H) +(0,0)     -- ($ (A') + (1,0) $);
\draw [red, dashed] (H) +(0,0.75)  -- (A');
\draw [red, dashed] (H) +(0,-0.75) -- (A');

\node [below]      at (A)  {$\point{A}$};
\node [below]      at (A') {$\point{A'}$};
\node [above left] at (H) {$\point{H}$};
\end{tikzpicture}
\end{center}
\end{figure}

\noindent La \imp{relation de conjugaison du miroir plan} s'écrit alors :
\[\algebrique{\point{A}\point{H}} + \algebrique{\point{A'}\point{H}} = 0\]
\end{propriete}

\begin{propriete}
Un \imp{objet plan} $\point{A}\point{B}$ parallèle au plan de $\mathscr{M}$ d'image $\point{A'}\point{B'}$ a pour \imp{grandissement transversal} :
\[\gamma = \frac{\algebrique{\point{A'}\point{B'}}}{\algebrique{\point{A}\point{B}}} = 1\]
\end{propriete}



\subsection{Dioptre plan}

\begin{propriete}
Un \imp{dioptre plan} $\mathscr{D}$ n'est pas rigoureusement stigmatique (et donc pas rigoureusement aplanétique) :

\begin{figure}[H]
\begin{center}
\begin{tikzpicture}[thick, use optics]
\coordinate (A) at (-1,0);

\foreach \x in {-60,-40,...,60}
{
	\draw [red, ->-] ($ (A) $) -- +(1,{tan(\x)});
	\draw [red, ->-] ($ (A) + (1,{tan(\x)}) $) -- +(1,{tan(asin(sin(\x)/2))});
	\draw [red, dashed] ($ (A) + (1,{tan(\x)}) $) -- +(-4,{-4*tan(asin(sin(\x)/2))});
}

\draw (0,-2.5) -- (0,2.5) coordinate (haut);
\node [left=0.2]  at (haut) {$n_1$};
\node [right=0.2] at (haut) {$n_2$};

\node [below] at (A) {$\point{A}$};
\end{tikzpicture}
\end{center}
\end{figure}
\end{propriete}

\begin{propriete} Les \imp{conditions de Gauss} permettent un stigmatisme approché :

\begin{figure}[H]
\begin{center}
\begin{tikzpicture}[thick, use optics]
\coordinate (A) at (-1,0);
\coordinate (H) at (0,0);

\foreach \x in {-40,-20,...,40}
{
	\draw [red, ->-] ($ (O) + (A) $) -- +(1,{tan(\x)});
	\draw [red, ->-] ($ (O) + (A) + (1,{tan(\x)}) $) -- +(1,{tan(asin(sin(\x)/2))});
	\draw [red, dashed] ($ (O) + (A) + (1,{tan(\x)}) $) -- +(-4,{-4*tan(asin(sin(\x)/2))});
}

\draw (0,-2.5) -- (0,2.5) coordinate (haut);
\node [slit, object height=4cm, slit height=1.8cm] at (-0.2,0) {};


\node [left=0.2]  at (haut) {$n_1$};
\node [right=0.2] at (haut) {$n_2$};

\node [below] at (A) {$\point{A}$};
\node [below] at (-2.4,0) {$\point{A'}$};
\node [above left] at (H) {$\point{H}$};
\end{tikzpicture}
\end{center}
\end{figure}

\noindent La relation de conjugaison du dioptre plan s'écrit alors :
\[\frac{n_1}{\algebrique{\point{A}\point{H}}} = \frac{n_2}{\algebrique{\point{A'}\point{H}}}\]
\end{propriete}