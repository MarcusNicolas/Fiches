\subsection{Lentilles de même axe optique}

\begin{propriete}
Si un système optique $\mathscr{S}$ est constitué de $n$ \abr{lms} $\mathscr{L}_1, \ldots, \mathscr{L}_n$ de même axe optique (placées dans cet ordre dans le sens de la lumière incidente), on construit l'image $\point{A'}$ d'un \imp{point objet} $\point{A}$ par $\mathscr{S}$ en construisant les images intermédiaires $\point{A}_1, \ldots, \point{A}_{n-1}$ par les différentes lentilles :
\[\point{A} \stackrel{\mathscr{L}_1}{\longmapsto} \point{A}_1 \stackrel{\mathscr{L}_2}{\longmapsto} \point{A}_2 \stackrel{\mathscr{L}_3}{\longmapsto} \ldots \stackrel{\mathscr{L}_n}{\longmapsto} \point{A'}\]
\end{propriete}

\begin{definition}
Le \tdef{grossissement} optique correspond au rapport entre l’angle $\alpha'$ sous lequel est vue l’image formée par le système optique et l’angle $\alpha$ sous lequel est vu l’objet (par rapport à l'axe optique) :
\[G = \frac{\alpha'}{\alpha}\] 
\end{definition}

\begin{vocabulaire}
On dit que l'image est :
\begin{itemize}
\item \tdef{plus grosse} (resp. \tdef{moins grosse}) que l'objet si $\abs{G} > 1$ (resp. $\abs{G} < 1$);
\item \tdef{droite} (resp. \tdef{renversée}) si $G > 0$ (resp. $G < 0$).
\end{itemize}
\end{vocabulaire}

\begin{exemple}
Une \tdef{lunette de visée à l'infini} sert à observer un objet à l'infini. Pour que l'observation se fasse sans effort (pour un \oe{}il emmétrope), il faut que l'image de cet objet par la lentille soit rejetée à l'infini. C'est donc un système afocal :
\[\point{A}_{\infty} \in \droite{\Delta} \stackrel[\text{objectif}]{\mathscr{L}_1}{\longmapsto} \point{F'}_1 = \point{F}_2 \stackrel[\text{oculaire}]{\mathscr{L}_2}{\longmapsto} \point{A'}_{\infty} \in \droite{\Delta}\]

\begin{figure}[H]
\begin{center}
\begin{tikzpicture}[thick, use optics]
\draw (-4,0) -- (4,0) node [right] {$\droite{\Delta}$};

\node[lens, lens type=converging, object height=3.5cm] (L1) at (-2,0) {};
\node [below] at (L1.south) {$\mathscr{L}_1$};
\node [below right] at (L1.center) {$\point{O}_1$};

\node[lens, lens type=converging, object height=3.5cm] (L2) at (1,0) {};
\node [below] at (L2.south) {$\mathscr{L}_2$};
\node [below right] at (L2.center) {$\point{O}_2$};

\draw [blue, ->-] (-3.5,-0.5) -- (-2,0);
\draw [blue, ->-] (-3.5,-1.5) -- (-2,-1);

\draw [blue, ->-] (-2,0)    -- (1,1);
\draw [blue, ->-] (-2,-1)   -- (1,1.5);

\draw [blue, ->-] (1,1) -- (3.5,-0.67);
\draw [blue, ->-] (1,1.5) -- (3.5,-0.17);

\draw [red, ->-] (-3.5,0.5) -- (-2,0);
\draw [red, ->-] (-3.5,1.5) -- (-2,1);

\draw [red, ->-] (-2,0)    -- (1,-1);
\draw [red, ->-] (-2,1)   -- (1,-1.5);

\draw [red, ->-] (1,-1) -- (3.5,0.67);
\draw [red, ->-] (1,-1.5) -- (3.5,0.17);

\draw [->] (-3.2,0) arc (180:161.57:1.2) node [midway, left] {$\alpha$};
\draw [->] (2,0) arc (180:213.69:0.5) node [midway, left] {$\alpha'$};

\draw [dashed] (0,-1.75) -- (0,1.75);
\node [above right] at (0,0) {$\point{F'}_1$};
\node [below right] at (0,0) {$\point{F}_2$};

\node (+) at (2.5,1.7) {$+$};
%\draw [->] (2,1.4) -- node [above] (+) {$+$} +(0.6,0);
\draw [->]  ($ (+) + (0.4,0) $) arc (0:90:0.4);
\draw [<->] ($ (+) + (-0.4,0.4) $) -- ++(0,-0.8) -- +(0.8,0);
\end{tikzpicture}
\end{center}
\end{figure}

\noindent Le grossissement de l'objet observé s'exprime alors :
\[G = -\frac{{f'}_1}{{f'}_2}\]
\end{exemple}



\subsection{Lentilles accolées}

\begin{definition}
Deux \imp{\abr{lms}} sont dites \tdef{accolées} lorsque leurs centres optiques $\point{O}_1$ et $\point{O}_2$ sont confondus :
\[\point{O}_1 = \point{O}_2 = \point{O}\]
\end{definition}

\begin{theoreme}[Lentilles accolées]
Deux \imp{\abr{lms}} $\mathscr{L}_1$ et $\mathscr{L}_2$ \imp{accolées} de vergences $V_1$ et $V_2$ se comportent comme une \abr{lms} $\mathscr{L}$ de vergence $V$ telle que :
\[V = V_1 + V_2\]

\begin{figure}[H]
\begin{center}
\begin{tikzpicture}[thick, use optics]
\coordinate (O1) at (-3.25,0);
\coordinate (O2) at (-2.75,0);
\coordinate (origine) at (-3,0);

\coordinate (O) at (3,0);

\node at (0,0) {$\equiv$};

% Left

\draw (origine) +(-2,0) -- +(2,0) node [right] {$\droite{\Delta}$};

\node[lens, lens type=converging, object height=2.4cm] (L1) at (O1) {};
\node [below left] at (O1) {$\point{O}_1$};

\node[lens, lens type=converging, object height=2.4cm] (L2) at (O2) {};
\node [below right] at (O2) {$\point{O}_2$};

\node [below] at (L1.south) {$\mathscr{L}_1$};
\node [below] at (L2.south) {$\mathscr{L}_2$};

% Right

\draw (O) +(-2,0) -- +(2,0) node [right] {$\droite{\Delta}$};

\node[lens, lens type=converging, object height=2.4cm] (L) at (O) {};

\node [below left] at (O) {$\point{O}$};

\node [below] at (L.south) {$\mathscr{L}$};
\end{tikzpicture}
\end{center}
\end{figure}
\end{theoreme}