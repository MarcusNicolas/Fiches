\subsection{Échelle}

\begin{definition}
On appelle \tdef{libre parcours moyen}, noté $\ell_p$, la distance moyenne parcourue par les molécules d'un fluide entre deux chocs successifs.
\end{definition}

\begin{remarque}
L'échelle de la thermodynamique est \tdef{l'échelle mésoscopique} qui est entre l'échelle microscopique ($\simeq \ell_p$) et l'échelle macroscopique. On peut alors parler de grandeurs définies à partir de moyennes statistiques (température, pression, \ldots) de volumes quasi-ponctuels.
\end{remarque}



\subsection{Système thermodynamique}

\begin{definition}
Un \tdef{système} est un corps ou un ensemble de corps séparés du \imp{milieu extérieur} par une \tdef{frontière} (qui peut être fictive). La réunion d'un système et de son milieu extérieur constitue l'\tdef{Univers}. Un \tdef{système thermodynamique} possède trop de particules pour les décrire individuellement.
\end{definition}

\begin{definitions}
Un \imp{système} est dit :
\begin{itemize}
\item \tdef{isolé} s'il ne peut échanger ni matière ni énergie avec le mileu extérieur;
\item \tdef{fermé} s'il ne peut échanger que de l'énergie (pas d'échange de matière);
\item \tdef{ouvert} s'il peut échanger matière et énergie.
\end{itemize}
\end{definitions}

\begin{propriete}[admis]
L'état macroscopique d'un \imp{système thermodynamique} peut être décrit à l'aide d'un petit nombre de grandeurs macroscopiques, appelées \tdef{grandeurs d'état}. On appelle \tdef{variables d'état} un ensemble de \imp{grandeurs d'état indépendantes} choisies pour caractériser le système.
\end{propriete}

\begin{propriete}[admis]
Les \imp{grandeurs d'état} qui ne sont pas des \imp{variables d'état} (appelées \tdef{fonctions d'état}) peuvent s'en déduire par une relation appelée \tdef{équation d'état}.
\end{propriete}

\begin{definition}
Un \tdef{fluide thermoélastique} est un \imp{système} caractérisé par les \imp{grandeurs d'état} suivantes :
\begin{itemize}
\item la pression $P$;
\item le volume $V$;
\item la température (absolue) $T$;
\item la quantité de matière $n$.
\end{itemize}
\end{definition}



\subsection{Extensivité et intensivité}

\begin{definitions}
Une \imp{grandeur d'état} caractérisant un système de quantité de matière $n$ est dite :
\begin{itemize}
\item \tdef{extensive} si elle est \imp{proportionnelle} à $n$;
\item \tdef{intensive} si elle \imp{indépendante} de $n$.
\end{itemize}
\end{definitions}

\begin{exemples}
Les grandeurs suivantes sont \imp{extensives} :
\begin{itemize}
\item la quantité de matière $n$;
\item la masse $m$;
\item le volume $V$.
\end{itemize}

\noindent Les suivantes sont \imp{intensives} :
\begin{itemize}
\item la température $T$;
\item la pression $P$.
\end{itemize}
\end{exemples}

\begin{remarque}
Certaines grandeurs ne sont ni intensives ni extensives (la grandeur $nm$ par exemple est proportionnelle à $n^2$).
\end{remarque}

\begin{propriete}
Une \imp{grandeur extensive} $X$ est \tdef{additive} : pour deux systèmes disjoints $\systeme{\Sigma}_1$ et $\systeme{\Sigma}_2$ on a :
\[X_{\systeme{\Sigma}_1 \cup \systeme{\Sigma}_2} = X_{\systeme{\Sigma}_1} + X_{\systeme{\Sigma}_2}\]
\end{propriete}

\begin{propriete}[admis]
Les \imp{grandeurs intensives} sont définies localement.
\end{propriete}

\begin{vocabulaire}
Un \imp{système} dans lequel les \imp{grandeurs intensives} sont définies et ont même valeur en tout point est dit \tdef{homogène}.
\end{vocabulaire}

\begin{propriete}
Le rapport de deux \imp{grandeurs extensives} est \imp{intensive}.
\end{propriete}