\subsection{Variance}

\begin{definition}
On appelle \tdef{corps pur} tout \imp{système} constitué d'uniquement une espèce chimique.
\end{definition}

\begin{definition}
Une \tdef{phase} est un domaine d'un système où les \imp{grandeurs intensives} sont des \imp{fonctions continues des variables d'espace}.
\end{definition}

\begin{definition}
La \tdef{variance} $v$ est le nombre minimal de \imp{paramètres intensifs} nécessaires pour caractériser l'état d'équilibre d'un système sous $\varphi$ phases.
\end{definition}

\begin{propriete}[admis]
La \imp{variance} d'un \imp{corps pur} à l'\imp{équilibre} sous $\varphi$ phases est donnée par la \tdef{règle des phases de Gibbs} :
\[v = 3 - \varphi\]
\end{propriete}

\begin{remarque}
En particulier, un \tdef{corps pur diphasé} (en équilibre sous deux phases) est un système \tdef{monovariant} ($v = 1$).
\end{remarque}



\subsection{Diagramme de phase}

\begin{definition}
Un \tdef{diagramme de phase} permet, pour un corps pur donné, de visualiser la dépendance $P = f(T)$ sur les courbes d'équilibre :

\begin{enumerate}[label=(\arabic*)]
\item la \tdef{courbe de vaporisation} (liquide/gaz) d'équation $P = P_{\mathrm{sat}}(T)$ où $P_{\mathrm{sat}}(T)$ est appelée \tdef{pression de vapeur saturante}. La courbe de vaporisation se termine au point $\point{C}$, appelé \tdef{point critique}.
\item la \tdef{courbe de fusion} (liquide/solide) est quasi-verticale, de pente positive pour la plupart des corps mais négative pour l'eau.
\item la \tdef{courbe de sublimation} (solide/gaz).
\end{enumerate}

\begin{figure}[H]
\begin{center}
\begin{tikzpicture}[thick]
\pgfmathsetmacro{\PT}{22.6};
\pgfmathsetmacro{\TT}{223};
\pgfmathsetmacro{\logPT}{log10(\PT)};
\pgfmathsetmacro{\logTT}{log10(\TT)};
\pgfmathsetmacro{\alpha}{10000};
\pgfmathsetmacro{\beta}{4};
\pgfmathsetmacro{\gamma}{1};

\coordinate (T) at (\logTT,\logPT);
\node at (T) {$\bullet$};
\node [below] at (T) {$\point{T}$};

% Liquide/solide
\draw [samples=200, domain=\logTT:{\logTT + 0.12}, variable=\logT] plot (\logT, {log10(\PT + \alpha * (\logT - \logTT))});

% Liquide/Gaz
\draw [samples=200, domain=\logTT:4, variable=\logT] plot (\logT, {\logPT - \gamma * (10^(-\logT) - 1/\TT))});

% Solide/Gaz
\draw [samples=200, domain=0.5:\logTT, variable=\logT] plot (\logT, {\logPT - \beta * (10^(-\logT) - 1/\TT))});

\draw [<->] (0,3) node [above] {$P$} -- (0,0) -- (4.5,0) node [right] {$T$};
\end{tikzpicture}
\end{center}
\end{figure}

\noindent Elles délimitent les domaines d'existence du corps pur sous l'une ses de phases.
\end{definition}

\begin{remarque}
Au-delà du \imp{point critique $\point{C}$}, la distinction entre gaz et liquide est impossible : on parle alors d'\tdef{état fluide} ou de \tdef{fluide hypercritique}.
\end{remarque}

\begin{propriete}
Les courbes d'équilibre du \imp{diagramme $(P, T)$} s'intersectent en un unique point $\point{T}$, appelé \tdef{point triple}. Le corps pur y est en équilibre sous ses trois phases, donc le système est invariant (règle de Gibbs).
\end{propriete}

\begin{propriete}[admis]
$P_{\mathrm{sat}}$ est toujours une fonction croissante de $T$.
\end{propriete}