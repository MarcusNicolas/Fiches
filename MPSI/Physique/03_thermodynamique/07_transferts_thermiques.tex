\subsection{Transformations adiabatiques}

\begin{definition}
On appelle transformation \tdef{adiabatique} une \imp{transformation} au cours de laquelle il n'y a \imp{aucun transfert thermique} entre le système et le milieu extérieur. Selon le caractère infinitésimal ou non de la transformation, on a :
\[\delta Q = 0 \qquad \text{ou} \qquad Q = 0\]
\end{definition}

\begin{remarque}
En pratique , pour réaliser une transformation adiabatique, on peut soit :

\begin{itemize}
\item placer le système dans une enceinte \tdef{calorifugée} munie de parois \tdef{athermanes} (parois qui interdisent les transferts thermiques avec le milieu extérieur, au contraire de parois \tdef{diathermanes});

\item effectuer une transformation \imp{suffisamment rapide} pour que les transferts thermiques (toujours beaucoup plus lents que les échanges d'énergie par travail) n'aient pas eu le temps de se produire.
\end{itemize}
\end{remarque}

\begin{attention}
Une transformation adiabatique n'est pas nécessairement isotherme.
\end{attention}



\subsection{Calcul pratique}

\begin{remarque}
Le cas d'une transformation adiabatique mis à part, un transfert thermique est toujours calculé de \imp{manière indirecte} (à l'aide du \imp{premier principe}, voir plus tard).
\end{remarque}