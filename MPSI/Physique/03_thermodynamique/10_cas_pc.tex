\subsection{Énergie interne et enthalpie}

\begin{propriete}
Pour une \imp{\abr{pc} idéale}, on a :
\[C_V = C_P\]
Il est donc inutile de distinguer $C_V$ et $C_P$, on parle simplement de \tdef{capacité thermique}, notée $C$.
\end{propriete}

\begin{propriete}[admis]
L'\imp{énergie interne} $U^{\text{\abr{pc}}}$ et l'\imp{enthalpie} $H^{\text{\abr{pc}}}$ d'un \imp{\abr{pc} idéale} ne dépendent que de $T$. On a donc :
\[\dif U^{\text{\abr{pc}}} = C \dif T \qquad \text{et} \qquad \dif H^{\text{\abr{pc}}} = C \dif T\]
\end{propriete}

\begin{remarque}
La \imp{capacité thermique} d'une \imp{\abr{pc} idéale} peut généralement être considérée constante sur de larges gammes de températures.
\end{remarque}

\begin{propriete}
Pour un \imp{\abr{pc}} subissant une \imp{transformation finie} :
\[\Delta U^{\text{\abr{pc}}} = C \Delta T \qquad \text{et} \qquad \Delta H^{\text{\abr{pc}}} = C \Delta T\]
\end{propriete}



\subsection{Entropie}

\begin{propriete}
L'\imp{entropie} $S^{\text{\abr{pc}}}$ d'une \imp{\abr{pc} idéale} s'exprime :
\[\dif S^{\text{\abr{pc}}} = C \frac{\dif T}{T} \qquad \text{et} \qquad S^{\text{\abr{pc}}} = C \ln T + cte\]
\end{propriete}