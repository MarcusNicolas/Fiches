\documentclass[french]{book}

\usepackage[top=2cm, bottom=2cm, left=2cm, right=2cm]{geometry}

\usepackage[T1]{fontenc}
\usepackage[utf8]{inputenc}
\usepackage{babel}
\usepackage{pdflscape}
\usepackage{titlesec}
\usepackage{multicol}
\usepackage[inline]{enumitem}
\usepackage{amsmath, amsthm, amssymb}
\usepackage{stackrel}
\usepackage[squaren, Gray]{SIunits}
\usepackage{chemformula}
\usepackage{mathrsfs}
\usepackage{float}
\usepackage{array}
\usepackage{caption}
\usepackage{hyperref} % Liens dans la table des matières


% TikZ

\usepackage{tikz}
\usetikzlibrary{babel}
\usetikzlibrary{calc}
\usetikzlibrary{arrows}
\usetikzlibrary{patterns}
\usetikzlibrary{decorations.pathmorphing, decorations.markings}
\usetikzlibrary{positioning}
\usetikzlibrary{optics}
\usetikzlibrary{shapes.misc}


% "north east hatch" pattern
\makeatletter
\tikzset{% customization of pattern 
        hatch distance/.store in=\hatchdistance,
        hatch distance=5pt,
        hatch thickness/.store in=\hatchthickness,
        hatch thickness=5pt
        }
\pgfdeclarepatternformonly[\hatchdistance,\hatchthickness]{north east hatch}% name
    {\pgfqpoint{-1pt}{-1pt}}% below left
    {\pgfqpoint{\hatchdistance}{\hatchdistance}}% above right
    {\pgfpoint{\hatchdistance-1pt}{\hatchdistance-1pt}}%
    {
        \pgfsetcolor{\tikz@pattern@color}
        \pgfsetlinewidth{\hatchthickness}
        \pgfpathmoveto{\pgfqpoint{0pt}{0pt}}
        \pgfpathlineto{\pgfqpoint{\hatchdistance}{\hatchdistance}}
        \pgfusepath{stroke}
    }
\makeatother

\tikzset{>=stealth}

\newcommand*\circled[1]{\tikz[baseline=(char.base)]{
            \node[shape=circle,draw,inner sep=1pt] (char) {\footnotesize{#1}};}}

\tikzset{schema/.style={rounded corners=#1, fill=lightgray, draw, inner sep=0pt},
         schema/.default=4pt}
\tikzset{bati/.style={pattern=north east hatch, hatch distance=7pt, hatch thickness=1.2pt, preaction={fill=lightgray}}}
\tikzset{ressort/.style 2 args={decorate, decoration={coil,aspect=#1,segment length=#2,amplitude=3mm}}}


% Pour les spectres d'émission / d'absorption

\usepackage{pgf-spectra}


% Couverture

\ifpdf
	\usepackage{pdfcolmk}
\fi

\ifxetex
	\usepackage{fontspec}
\fi

\usepackage{url}
\usepackage{graphicx}
\usepackage{pifont}

\newlength{\drop}

\newcommand*{\mytitle}{\begin{titlepage}
\begingroup
\drop = 0.13\textheight
\centering
\vspace*{\drop}
{\Huge Physique (\textsc{mpsi})}\\[\baselineskip]
{\Huge\itshape Année scolaire 2017-2018}\\[3\baselineskip]
{\Large \textit{Cours de} \textsc{N. Tancrez}}\par
\vfill
{\begin{center}\includegraphics[scale=1.5]{SL.png}\end{center}}
\vspace*{1cm}
{\Large Lycée Saint-Louis}\par
\vspace*{\drop}
\endgroup
\end{titlepage}}



% Style des sections

\titleformat{\chapter}
[frame] % format
{\vspace*{5cm}\Huge} % format du texte
{\thechapter} % format du label
{1cm} % séparation
{\centering\sffamily} % avant
[\thispagestyle{empty}] % après


% pour la table des matières
\titleformat{name=\chapter, numberless}
[display] % format
{\Large\bfseries} % format du texte
{\titlerule} % format du label
{-7ex} % séparation
{\centering\MakeUppercase} % avant


\titleformat{\section}
[block] % format
{\LARGE\bfseries} % format du texte
{\thesection} % format du label
{8pt} % séparation
{\centering} % avant

\titleformat*{\subsection}{\Large\bfseries}


\renewcommand{\thechapter}{\Roman{chapter}}
\renewcommand{\thesection}{\arabic{section}.}
\renewcommand{\thesubsection}{\Roman{subsection}}

\pagestyle{plain} % Numéro de page en bas de la page
\setcounter{tocdepth}{1} % On ne veut que les sections

% Math environments

\newtheorem*{lemme}{Lemme}
\newtheorem*{theoreme}{Théorème}
\newtheorem*{propriete}{Propriété}
\newtheorem*{proprietes}{Propriétés}

\theoremstyle{definition}
\newtheorem*{definition}{Définition}
\newtheorem*{definitions}{Définitions}
\newtheorem*{principes}{Principes}
\newtheorem*{exemple}{Exemple}
\newtheorem*{exemples}{Exemples}
\newtheorem*{experience}{Expérience}
\newtheorem*{vocabulaire}{Vocabulaire}
\newtheorem*{methode}{Méthode}

\theoremstyle{remark}
\newtheorem*{remarque}{Remarque}


% Math macros

\usepackage{mathtools}

\DeclarePairedDelimiter\abs{\lvert}{\rvert}
 
\makeatletter
\let\oldabs\abs
\def\abs{\@ifstar{\oldabs}{\oldabs*}}
\makeatother


\newcommand*\dif{\mathop{}\!\mathrm{d}}


% Conventions du cours
\newcommand*{\avogadro}{\mathcal{N}_{\mathrm{A}}}

\newcommand*{\point}[1]{\mathrm{#1}}
\newcommand*{\droite}[1]{\mathrm{#1}}
\newcommand*{\systeme}[1]{\mathrm{#1}}
\newcommand*{\vecteur}[1]{\overrightarrow{#1}}
\newcommand*{\algebrique}[1]{\overline{#1}}


\newcommand*{\tdef}[1]{\textbf{#1}}
\newcommand*{\imp}[1]{\emph{#1}}
\newcommand*{\abr}[1]{\textsc{#1}}



% Où chercher les fichiers

\makeatletter
\def\input@path{{01_signaux_harmoniques_propagation/}, {02_optique_geometrique/}, {03_thermodynamique/}}
\makeatother

\newcommand{\cours}[2]
{\begin{landscape}
\begin{multicols*}{2}[\section{#1}]
\input{#2}
\end{multicols*}
\end{landscape}}


%%%%%%% Document %%%%%%%

\begin{document}

% Titre

\mytitle

\tableofcontents

\chapter{\textsc{Signaux harmoniques et propagation}}

\cours{Oscillateur harmonique}{01_oscillateur_harmonique}
\cours{Propagation d'un signal}{02_propagation_signal}
\cours{Ondes progressives sinusoïdales}{03_ondes_progressives_sinusoidales}
\cours{Interférences}{04_interferences}
\cours{Ondes stationnaires}{05_ondes_stationnaires}
\cours{Diffraction}{06_diffraction}

% TODO : ordres de grandeur



\chapter{\textsc{Optique géométrique}}

\cours{Description ondulatoire de la lumière}{01_description_ondulatoire_lumiere}
\cours{Modèle géométrique de la lumière}{02_modele_geometrique_lumiere}
\cours{Systèmes optiques}{03_systemes_optiques}
\cours{Systèmes centrés}{04_systemes_centres}
\cours{Foyers et plans focaux}{05_foyers_plans_focaux}
\cours{Miroir et dioptre plan}{06_miroir_dioptre}
\cours{Lentilles minces sphériques}{07_lentilles_minces_spheriques}
\cours{\abr{lms} dans les conditions de Gauss}{08_lms_conditions_gauss}
\cours{Association de \abr{lms}}{09_association_lms}



\chapter{\textsc{Thermodynamique}}

\cours{Description d'un système thermodynamique}{01_description_systeme_thermodynamique}
\cours{Transformation thermodynamique}{02_transformation_thermodynamique}
\cours{États de la matière}{03_etats_matiere}
\cours{Corps purs}{04_corps_purs}
\cours{Corps purs diphasés}{05_corps_purs_diphases}
\cours{Grandeurs thermodynamiques}{06_grandeurs_thermodynamiques}
\cours{Cas du \abr{gp}}{07_cas_gp}
\cours{Cas de la \abr{pc} idéale}{08_cas_pc}


\begin{landscape}
\begin{multicols*}{2}[\section{Travail des forces non conservatives}]

\subsection{Travail des forces de pression}

\begin{propriete}
Le \tdef{travail élémentaire des forces de pression} s'exerçant sur un système thermodynamique est égal à :
\[\delta W_P = -P_{\mathrm{front}} \dif V\]
\end{propriete}

\begin{remarque}
Pour une transformation \imp{\abr{qs}}, l'équilibre mécanique implique que $P = P_{\mathrm{front}}$. Le travail des forces de pression s'exprime alors :
\[\delta W_P = -P \dif V \quad \text{soit} \quad W_{P, \point{I} \to \point{F}} = \int_{V_{\point{I}}}^{V_{\point{F}}} -P \dif V\]
\end{remarque}

\begin{propriete}

Dans un diagramme de Watt ($P$,$V$) ou de Clapeyron :
Si $V_I<V_F$ (resp. $V_I>V_F$), dans le cas d'une \imp{détente} (resp. \imp{compression}, l'aire sous la courbe est égale à l'opposé du travail (resp. au travail) $W^{nc}_{I\rightarrow F}$ reçu par le système :
\[
\mathscr{A} = \int_{V_I}^{V_F}PdV = -W^{nc}_{I\rightarrow F} \text{(resp. } \mathscr{A} = \int_{V_F}^{V_I}PdV = W^{nc}_{I\rightarrow F} \text{)}
\]
Ainsi, le travail reçu par le système est négatif (resp. positif). Il cède (resp. reçoit) de l'énergie sous forme de travail.

\end{propriete}

\begin{propriete}

$W_{cycle} = -\mathscr{A}_{cycle} < 0$ (resp. $= \mathscr{A}_{cycle} > 0$)  si le cycle est parcouru dans le sens des aiguilles d'une montre (resp. le sens inverse) : on dit que le cycle est \tdef{moteur} (resp. \tdef{récepteur}).

\end{propriete}

\begin{remarque}

En pratique, la propriété précédente pourra être utilisée dans le cas ù les transformations cycliques étudiées correspondront à des polygones dans un diagramme ($P$,$V$).

\end{remarque}

\begin{propriete}[$W_P$ pour des transformations usuelles]

\begin{itemize}
\item Transformation isochore : $\delta W_P = 0 et W_P = 0$
\item Transformation monobare à $P_0$ : $W_P = -P_0(V_F - V_I)$
\item Transformation \abr{qs} isotherme à $T_0$ d'un \abr{gp} à $n$ constant : $W_P = -nRT_0\ln \left( \frac{V_F}{V_I} \right) $
\end{itemize}

\end{propriete}

\begin{propriete}

Le travail électrique élémentaire $\delta W_{élec}$ reçu entre les dates $t$ et $t + dt$ par un dipôle parcouru par un courant d'intensité $i(t)$ et soumis à la tension $u(t)$ en convention récepteur est :
\[
\delta W_{élec} = u(t)i(t)dt
\]

\end{propriete}

\end{multicols*}
\end{landscape}



\end{document}
